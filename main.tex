% Comandos para crear la portada y la primera página del PFG
%% Título del PFG
\newcommand{\PFGtitulo}{Virtualización ligera y cloud computing para el despliegue de servidores streaming para la docencia online de ingeniería.}
%% Autor/a y directores/as del PFG
\newcommand{\PFGautor}{Ángel Roberto García Serpa}
\newcommand{\PFGdirector}{Agustín Carlos Caminero Herráez}
%% Descomentar la siguiente línea si hay codirector/a en el PFG
%\newcommand{\PFGcodirector}{Codirector/a del PFG}
%% Grado. A elegir entre:
%% Ingeniería Informática
%% Ingeniería en Tecnologías de la Información
\newcommand{\PFGgrado}{Ingeniería Informática}
%% Modalidad. A elegir entre:
%% genérica
%% específica
\newcommand{\PFGmodalidad}{genérica}
%% Curso académico en formato 20xx/20xx+1. Por ejemplo 2022/2023
\newcommand{\PFGcursoacademico}{2025/2026}
%% Convocatoria. A elegir entre:
%% junio
%% septiembre
%% diciembre
\newcommand{\PFGconvocatoria}{diciembre}
%% Fecha de lectura (si no se sabe el día concreto, poner mes y año}
\newcommand{\PFGfechadefensa}{Fecha de lectura}


\documentclass[12pt]{book}
\usepackage{graphicx}
\usepackage[utf8]{inputenc}
\usepackage{indentfirst}% This package sets the boolean \@afterindentfalse to (always) true
\usepackage[T1]{fontenc}
%\usepackage{tgbonum}
\usepackage[papersize={210mm,297mm},tmargin=25mm,bmargin=25mm,lmargin=20mm,rmargin=20mm]{geometry}

% no indentar los párrafos
\setlength\parindent{0cm}
\setlength\parskip{0.2em}

%Para tener los elementos de texto en español
\usepackage[spanish]{babel}
%Hipervinculos para todo lo útil
%\usepackage{hyperref}
\usepackage{xcolor}
\definecolor{links}{rgb}{0.35, 0.4, 0.35}
\definecolor{urls}{rgb}{0,0,1}
\definecolor{cites}{rgb}{0.8,0,0}

\usepackage{hyperref}
%\usepackage[all]{hypcap}
\hypersetup{
	colorlinks = true,
	linkcolor = links,
	urlcolor = urls,
	citecolor = cites
}

\usepackage{fancyhdr}

%añadir ficheros externos
\usepackage{subfiles}
%soporte para anexos
\usepackage{appendix}
%paquetes para figuras
\usepackage{caption}
\usepackage{subcaption}
\usepackage{listings}
\usepackage{graphicx}
\usepackage[rightcaption]{sidecap}

%Encabezados bonitos
\pagestyle{headings}

%paquete para poner la bibliografia en la table of contents
\usepackage[nottoc,notlot,notlof]{tocbibind}

%Cambiamos el espacio en blanco enorme que por defecto hay en las páginas inicio de capítulo
\usepackage{titlesec}
\titleformat{\chapter}[display]   
{\normalfont\huge\bfseries}{\chaptertitlename\ \thechapter}{20pt}{\Huge}   
\titlespacing*{\chapter}{0pt}{-50pt}{40pt}

%para crear subsubsubsection
\setcounter{secnumdepth}{4}
\titleformat{\paragraph}
{\normalfont\normalsize\bfseries}{\theparagraph}{1em}{}
\titlespacing*{\paragraph}
{0pt}{3.25ex plus 1ex minus .2ex}{1.5ex plus .2ex}

%Ruta en la que pondremos las imágenes para el documento
\graphicspath{ {imagenes/} }
%Redifinimos el nombre que asigna babel spanish a las tablas (cuadro) a tabla
\renewcommand{\spanishtablename}{Tabla}
\renewcommand{\spanishlisttablename}{Índice de tablas}
\renewcommand{\spanishcontentsname}{Índice}
\renewcommand{\appendixname}{Anexos}
\renewcommand{\appendixtocname}{Anexos}
\renewcommand{\appendixpagename}{Anexos}

% Definimos el comando de página en blanco
\newcommand\paginablanco{%
    \null
    \thispagestyle{empty}%
    \newpage}

% Definimos el comando de página en blanco sin avanzar numeracion
\newcommand\paginablancosin{%
    \paginablanco{}
    \addtocounter{page}{-1}}

%mayor anchura en las tablas
\renewcommand{\arraystretch}{1.5}
%mayor ancuhura en las fracciones
\newcommand\ddfrac[2]{\frac{\displaystyle #1}{\displaystyle #2}}
%comando escribir TFG
\newcommand{\tfg}{Trabajo Fin de Grado }
%estilo de fuente para la consola
\usepackage{ascii}
\usepackage[T1]{fontenc}
%definimos estilo consola
\definecolor{mygreen}{rgb}{0,0.6,0}
\definecolor{mygray}{rgb}{0.5,0.5,0.5}
\definecolor{mymauve}{rgb}{0.58,0,0.82}
\definecolor{terminalbgcolor}{HTML}{000000}
\definecolor{terminalrulecolor}{HTML}{000099}
\newcommand{\lstconsolestyle}{
	\lstset{
		backgroundcolor=\color{terminalbgcolor},
		basicstyle=\color{white}\asciifamily\footnotesize\selectfont,
		breakatwhitespace=false,  
		breaklines=true,
		captionpos=b,
		commentstyle=\color{mygreen},
		deletekeywords={...},
		escapeinside={\%*}{*)},
		extendedchars=true,
		frame=single,
		keepspaces=true,
		keywordstyle=\color{blue},
		%language=none,
		morekeywords={*,...},
		numbers=none,
		numbersep=5pt,
		framerule=2pt,
		numberstyle=\color{mygray}\tiny\selectfont,
		rulecolor=\color{terminalrulecolor},
		showspaces=false,
		showstringspaces=false,
		showtabs=false,
		stepnumber=2,
		stringstyle=\color{mymauve},
		tabsize=2
	}
}

\newcommand{\lstnginxconfig}{
	\lstset{
		backgroundcolor=\color{black!5}, % Fondo gris claro para el bloque de código
		commentstyle=\color{gray!80}\ttfamily, % Comentarios en gris y monospace
		keywordstyle=\color{red!70!black}\bfseries, % Palabras clave (ej. user, worker_processes) en rojo oscuro y negrita
		stringstyle=\color{orange!80!black}\ttfamily, % Cadenas (si las hubiera, aunque aquí no son predominantes)
		basicstyle=\ttfamily\small, % Fuente básica monospace y pequeña
		numbers=left, % Números de línea a la izquierda
		numberstyle=\tiny\color{gray}, % Estilo de los números de línea
		frame=single, % Marco alrededor del código
		framesep=5pt, % Espaciado del marco
		framerule=0.5pt, % Grosor del marco
		rulesepcolor=\color{gray!20}, % Color de las líneas de separación
		breaklines=true, % Permite saltos de línea
		captionpos=b, % Posición del título (abajo)
		tabsize=4, % Tamaño de la tabulación
		showstringspaces=false, % No muestra espacios en blanco en las cadenas
		% Definición de palabras clave específicas de Nginx
		keywords={user, worker_processes, error_log, pid, events, worker_connections, rtmp, server, listen, chunk_size, application, live, record, hls, hls_path, hls_fragment, hls_playlist_length, hls_sync, dash, dash_path, dash_fragment, dash_playlist_length, play},
		morecomment=[l]{\#}, % Comentarios de una línea que empiezan con #
		morestring=[b]", % Cadenas entre comillas dobles
	}
}

% Comenzamos el documento
\begin{document}
\pagenumbering{gobble} % para no incluir números en la portada y primera página

% Hacemos la portada
\begin{titlepage}

\begin{figure}[!htb]
    \centering
    \includegraphics{rosadelosvientos.png}
\end{figure}

\begin{center}
    {\scshape\Large Universidad Nacional de Educación a Distancia \par}
    {\scshape\Large Escuela Técnica Superior de Ingeniería Informática \par}
    \vfill
    \large{Proyecto de fin de Grado en \PFGgrado}\\
    \vfill
    {\bfseries\Huge \textcolor{black}{\PFGtitulo} \par}
    \vfill
\end{center}

\vspace*{3cm}
{\huge \textcolor{black}{\PFGautor} \par}
{\huge \textcolor{black}{Dirigido por: \PFGdirector} \par}
\ifdefined\PFGcodirector
  \huge \textcolor{black}{Codirigido por: \PFGcodirector} \par
\fi
{\huge \textcolor{black}{Curso \PFGcursoacademico, convocatoria \PFGconvocatoria} \par}

\chapter*{}

\thispagestyle{empty}

\begin{figure}[!htb]
    \centering
    \includegraphics{rosadelosvientos.png}
\end{figure}

\begin{center}

    {\bfseries\Huge \textcolor{black}{\PFGtitulo} \par}
    \vfill
    {\bfseries\Large \textcolor{black}{Proyecto de fin de Grado en \PFGgrado} \par}
    {\bfseries\Large \textcolor{black}{de modalidad \PFGmodalidad} \par}
\end{center}

\vspace*{5cm}
{\Large Realizado por: \PFGautor \par}
{\Large Dirigido por: \PFGdirector \par}
\ifdefined\PFGcodirector
  {\Large Codirigido por: \PFGcodirector \par}
\fi%

%%{\Large Codirigido por: \PFGcodirector \par}
\vfill
{\Large Fecha de lectura y defensa: \PFGfechadefensa \par}

\end{titlepage}

\pagenumbering{Roman} % para comenzar la numeracion de paginas en numeros romanos

% Agradecimientos
\chapter*{Agradecimientos}
\thispagestyle{empty}
\begin{flushright}
\textit{Incluir todos los agradecimientos}
\end{flushright}

\chapter*{Resumen} % El * indica que no queremos que añada la palabra "Capitulo"

Este proyecto se centra en configurar varios servidores que sean capaces de manejar streaming de video sobre contenedores Docker. Esta configuración será tanto en local como en AWS . Otro aspecto clave es el uso de kubernetes para ayudar con el despliegue y escalado de las aplicaciones consiguiendo un manejo eficiente de los recursos en cada momento. Para esto último es necesaria la monitorización de los contenedores. 
Se han elegido dos tecnologías muy utilizadas en este momento en el streaming de video y las videollamdas, como son HLS y WebRTC.


\chapter*{Abstract} % El * indica que no queremos que añada la palabra "Capitulo"

This project focuses on configuring several video streaming servers using Docker containers. This configuration will be both in local and on AWS. Another key aspect is the use of kubernetes that is in charge of automating de deploying, scaling and management of the containerized applications. For achieving this last part it is necesary monitoring the applications.
Two technologies, such as HLS and WebRTC, were choosed for this Project because they are widely used at the moment in video streaming.

\chapter*{Palabras clave} % El * indica que no queremos que añada la palabra "Capitulo"

servidores de video, HLS, WebRTC, Docker, Kubernetes 

\chapter*{Keywords} % El * indica que no queremos que añada la palabra "Capitulo"

video servers, HLS, WebRTC, Docker, Kubernetes

\tableofcontents %% Índice general
\listoftables    %% Índice de tablas
\addcontentsline{toc}{section}{Índice de tablas} % Queremos que aparezca en el índice
\listoffigures   %% Índice de figuras
\addcontentsline{toc}{section}{Índice de figuras} % Queremos que aparezca en el índice
\newpage

\paginablanco{}

%% Comenzamos con la numeración arábica porque aquí empieza el contenido
\pagenumbering{arabic}

%% Ejemplo de importación de capítulos
\chapter{Introducción}
\subfile{capitulos/C1-introduccion}

\chapter{Estado del arte}
\subfile{capitulos/C2-estado_arte}
\chapter{Propuesta}
\subfile{capitulos/C3-propuesta}
\chapter{Diseño}
\subfile{capitulos/C4-disenno}
\chapter{Desarrollo del proyecto}
\subfile{capitulos/C5-desarrollo}
\chapter{Pruebas}
\subfile{capitulos/C6-pruebas}
\chapter{Conclusiones}
\subfile{capitulos/C7-conclusiones}

%% Bibliografía
\newpage
\subfile{capitulos/referencias}

\clearpage

\appendix
\chapter{}
\subfile{capitulos/A1-anexo}

\chapter{}
\subfile{capitulos/A2-anexo}

%...

\end{document}