\documentclass[../main.tex]{subfiles}

\begin{document}

1. Streaming de vídeo: fundamentos y tecnologías actuales

Codificación y codecs modernos: formatos como AV1, VP9 o H.264, sus ventajas respecto a calidad y eficiencia energética
en.wikipedia.org+2es.wikipedia.org+2es.wikipedia.org+2
.

Adaptive Bitrate Streaming (ABR): protocolos como MPEG‑DASH, HLS o Smooth Streaming, y cómo funcionan los algoritmos del cliente para ajustar calidad según ancho de banda
es.wikipedia.org+1en.wikipedia.org+1
.

Arquitecturas de extremo a extremo: revisión holística del pipeline de streaming, incluyendo captura, ingesta, transcodificación, entrega (CDN) y reproducción cliente
arxiv.org
.

Edge \& MEC (Mobile Edge Computing): almacenamiento y procesamiento en nodos cerca del usuario para mejorar latencia y eficiencia, uso de IA para caching y offloading
rua.ua.es+7arxiv.org+7wpd.ugr.es+7
.

Modelos híbridos P2P‑CDN: arquitectura híbrida que mejora la latencia, el consumo energético y la calidad para streaming en vivo
arxiv.org
.

Codificación adaptativa energéticamente eficiente: estrategias como LADRE con algoritmos ML para balancear calidad, latencia y consumo energético
arxiv.org
.

2. Contenerización y orquestación de servidores de contenidos

Virtualización basada en contenedores: diferencias clave entre contenedores y máquinas virtuales (recursos, arranque, aislamiento, portabilidad)
es.wikipedia.org
.

Uso práctico en broadcast y streaming: microservicios, contenedorización y orquestación en infraestructuras de radiodifusión modernas, con el fin de mejorar la flexibilidad y resiliencia operativa
evs.com
.

Servidor de streaming de baja latencia en contenedor: OvenMediaEngine (OME), proyecto open‑source con Docker, optimizado para streaming de baja latencia en arquitecturas contenedorizadas
en.wikipedia.org
.

Monitoreo y registro en pipelines contenedorizados: prácticas clave en DevOps para asegurar rendimiento, detección de errores y optimización de servicios despliegues en contenedores
fastercapital.com
.

3. Estructura sugerida para tu estado del arte

Una posible organización de tu documento:

Introducción histórica: evolución desde los codecs iniciales hasta sistemas actuales de streaming .

Tecnologías de streaming adaptativo: ABR, códecs eficientes, estándares y casos reales.

Arquitecturas modernas de entrega: CDN, MEC, sistemas híbridos, tendencias energéticas y sostenibles.

Contenerización de servidores de contenido: cómo Docker y Kubernetes potencian despliegues escalables y modulares.

Microservicios y orquestación: ventajas operativas, automatización CI/CD y escalado en broadcast
es.wikipedia.org+1en.wikipedia.org+1
.

Caso de estudio o solución real: por ejemplo, OME + Docker + métricas QoE
scielo.org.co+7riunet.upv.es+7en.wikipedia.org+7
.

Retos actuales y líneas futuras: latencia ultra‑baja, uso de IA, green streaming, seguridad en contenedores.

4. Recursos recomendados para lectura e investigación

“An End‑to‑End Pipeline Perspective on Video Streaming” (marzo 2024): encuesta detallada del pipeline completo con más de 200 artículos analizados
linkedin.com+2arxiv.org+2arxiv.org+2
.

“A Survey on Mobile Edge Computing for Video Streaming” (septiembre 2022): MEC como paradigma emergente para streaming eficiente en red 5G
arxiv.org
.

“Towards Low‑Latency and Energy‑Efficient Hybrid P2P‑CDN Live Video Streaming” (marzo 2024): propuesta híbrida innovadora para servicios de streaming en vivo
arxiv.org
.

“Energy‑efficient Adaptive Video Streaming with Latency‑Aware Dynamic Resolution Encoding (LADRE)” (enero 2024): uso de ML para mejorar eficiencia energética en encoding
arxiv.org
.

Blog EVS (febrero 2025) acerca de contenerización en broadcasting: microservicios, contenedores y orquestación
evs.com
.

Web oficial de OvenMediaEngine: documentación técnica y compatibilidad con Docker para streaming bajo contenedor
en.wikipedia.org
.

5. Enlaces útiles para consulta

Aquí tienes los recursos agrupados para que accedas rápidamente:
Tema	Recurso
Pipeline completo streaming	An End‑to‑End Pipeline Perspective
kasmweb.com
MEC aplicado al vídeo	A Survey on Mobile Edge Computing…
Arquitectura híbrida P2P‑CDN	Towards Low‑Latency Hybrid P2P‑CDN…
Codificación eficiente por ML	Energy‑efficient Adaptive Video Streaming…
Contenerización en broadcasting	Blog EVS: Adopting Software Containerization…
Servidor de streaming open‑source	OvenMediaEngine y Docker 

\end{document}