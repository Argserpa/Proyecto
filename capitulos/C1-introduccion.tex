\documentclass[../main.tex]{subfiles}

\begin{document}

\paragraph*{}
La utilización de "nuevas tecnologías" para la educación es uno de los temas centrales de las universidades y centros educativos en todo el mundo. La aparición de ARPANET a finales de la década de los 60 inicia la carrera para la democratización del conocimiento poniendo a disposición de todo aquel que tenga un dispositivo capaz de conectarse a internet información sobre casi cada tema que una persona pueda encontrar interesante. 
El siguiente paso, y salto a partir de la pandemia de 2021 ha sido el de impartir clases de forma remota. Muchas universidades empezaron a utilizar las tecnologías de la información como una herramienta más a la hora de impartir clases. \textbf{Revisar}
\paragraph*{} La UNED ha conseguido acercar la experiencia de la formación presencial a la modalidad remota mediante la utilización de herramientas que se han ido modernizando con el paso del tiempo. Desde el uso de la radio y vídeos, pasando por televisión y más recientemente las clases online, la UNED siempre ha intentado poner a disposición de sus alumnos material audiovisual que ayudara al alumnado a la obtención de los conocimientos necesarios. Actualmente la utilización de diversas plataformas online ayudan a conseguir dicho objetivo. Una de las maneras en que se puede mejorar la calidad de la educación a distancia es mediante las clases online y el streaming de vídeo online. En este momento esta es una de las herramientas utilizadas por nuestra universidad y por ello es interesante conocer herramientas que puedan ayudar a mejorar la calidad de las enseñanzas.
\paragraph*{}
En este TFG he intentado revisar algunas tecnologías que ayuden a realizar el lema de nuestra UNED "Que la sabiduría se mueva más que las cosas que se mueven". Para ello he llevado a cabo pruebas de renidmiento sobre servidores de vídeo que puedan ayudar a realizar eficientemente streaming de vídeo y que esto se convierta en una herramienta más que ayude a la transferencia de información del profesorado al alumnado.

\paragraph*{}
Voy a revisar tecnologías que se utilizan para el streaming de video como son servidores de emisión de video unidireccionales que se han utilizado tradicionalmente codificando señales de video enviadas y servidas en formatos que puedan consumirse desde un navegador web o un reproductor de video. Tambień servidores web basados en webRTC que mejoran la latencia de los anteriores y cuya comunicación bidireccional hace que se pueda dar conversación entre los participantes.

\end{document}