\documentclass[../main.tex]{subfiles}

\begin{document}
%
%\section{opción A}
%
%Se seguira el libro Mastering kubernetes \textbf{Revisar ToDo} en su cuarta edicion. Para la creacion del cluster es necesario instalar Rancher desktop
%\href{https://docs.rancherdesktop.io/getting-started/installation/\#linux}{Instalar Rancher}. Ésta es una aplicacion de escritorio que permitira ejecutar Docker. Rancher Desktop instalara otras herramientas como: 
%
%\begin{itemize}
%	\item 	Helm
%	\item 	Kubectl
%	\item 	Nerdctl
%	\item 	Moby (open source Docker)
%	\item 	Docker Compose
%\end{itemize}
%
%Se instala pass, necesario para guardar credenciales que se pasan via docker login y nerdctl login.
%
%Una vez instalado Rancher es recomendable generar una GPG key para securizar los secrets. para ello hacmoes lo siguiente:
%\begin{figure}[h!]
%	\begin{minipage}{0.7\textwidth}
%		\centering
%		\includegraphics[width=\textwidth]{pass00.png}
%		\caption{Generar gpg key}
%	\end{minipage}
%	\label{fig:figure0001}
%\end{figure}
%
%\begin{figure}[h!]
%	\begin{minipage}{0.7\textwidth}
%		\centering
%		\includegraphics[width=\textwidth]{rancher00.png}
%		\caption{Rancher}
%	\end{minipage}
%	\label{fig:figure0002}
%\end{figure}
%
%
%Se instala docker desktop siguiendo las instrucciones de la página oficial
%\href{https://docs.docker.com/engine/install/ubuntu/#install-using-the-repository}{Instalar Docker} 
%
%comprobamos que se han instalado correctamente tanto kubectl como docker
%\begin{figure}[h!]
%	\begin{minipage}{0.7\textwidth}
%		\centering
%		\includegraphics[width=\textwidth]{dockerKubectlVerions.png}
%		\caption{versiones de Docker y kubernetes}
%	\end{minipage}
%	\label{fig:figure0003}
%\end{figure}
%
%\begin{figure}[h!]
%	\begin{minipage}{0.7\textwidth}
%		\centering
%		\includegraphics[width=\textwidth]{docker00.png}
%		\caption{Docker}
%	\end{minipage}
%	\label{fig:figure0004}
%\end{figure}
%
%el usuario debe tener permisos de docker y de video 
%\begin{lstlisting}
%  sudo usermod -aG video $USER
%    sudo usermod -aG docker $USER
%\end{lstlisting}    
    
%\pagebreak
%\section{opción B, de momento sigue esta}
Se utiliza una distribución basada en Ubuntu como es Mint en su versión XFCE que es una versión ligera y se realiza la instalación en una MV para comprobar que la instalación funciona
se deb instalar: 
Docker Engine:
Siguiendo las instrucciones de la Documentación de Docker:
\href{https://docs.docker.com/engine/install/ubuntu/#install-using-the-repository}{Instalar Docker Engine} 

\href{https://docs.docker.com/engine/install/ubuntu/#install-using-the-repository}{Instalar Docker Desktop} 


\begin{figure}[h!]
	\begin{minipage}{0.7\textwidth}
		\centering
		\includegraphics[width=\textwidth]{000Docker.png}
		\caption{Instalacion Docker:primeros pasos.}
	\end{minipage}	
	\label{fig:figure001}
\end{figure}


\begin{figure}[h!]
	\begin{minipage}{0.7\textwidth}
		\centering
		\includegraphics[width=\textwidth]{001Docker.png}
		\caption{Instalacion Docker: Instalar paquetes de Docker.}
	\end{minipage}
	\label{fig:figure002}
\end{figure}
\begin{figure}[h!]
	\begin{minipage}{0.7\textwidth}
		\centering
		\includegraphics[width=\textwidth]{002Docker.png}
		\caption{Instalacion Docker: comprobar que funciona correctamente.}
	\end{minipage}		
	\label{fig:figure003}
\end{figure}

\begin{figure}[h!]
	\begin{minipage}{0.7\textwidth}
		\centering
		\includegraphics[width=\textwidth]{003Docker.png}
		\caption{Instalacion Docker: usuario }
	\end{minipage}		
	\label{fig:figure004}
\end{figure}

Añadimos el usuario al grupo de Docker y probamos que se puede ejecutar sin sudo:

https://docs.docker.com/desktop/setup/install/linux/ubuntu/
\begin{figure}[h!]
	\begin{minipage}{0.7\textwidth}
		\centering
		\includegraphics[width=\textwidth]{004Docker.png}
		\caption{Instalacion Docker}
	\end{minipage}
	\label{fig:figure005}
\end{figure}



kubectl
\href{https://kubernetes.io/docs/tasks/tools/install-kubectl-linux/#install-kubectl-binary-with-curl-on-linux}{Instalar Kubectl}.
\begin{figure}[h!]
	\begin{minipage}{0.7\textwidth}
		\centering
		\includegraphics[width=\textwidth]{005Kubectl.png}
		\caption{Instalacion Kubectl}
	\end{minipage}
	\label{fig:figure006}
\end{figure}

\begin{figure}[h!]
	\begin{minipage}{0.7\textwidth}
		\centering
		\includegraphics[width=\textwidth]{006Kubectl.png}
		\caption{Instalacion Kubectl}
	\end{minipage}
	\label{fig:figure007}
\end{figure}

\begin{figure}[h!]
	\begin{minipage}{0.7\textwidth}
		\centering
		\includegraphics[width=\textwidth]{007Kubectl.png}
		\caption{Instalacion Kubectl}
	\end{minipage}
	\label{fig:figure008}
\end{figure}


Instalamos y comprobamos versión de docker compose
\begin{figure}[h!]
	\begin{minipage}{0.7\textwidth}
		\centering
		\includegraphics[width=\textwidth]{008DockerCompose.png}
		\caption{Instalacion Docker Compose}
	\end{minipage}
	\label{fig:figure009}
\end{figure}

Para poder utilizar Docker Desktop en una MV se debe configurar en VirtualBox:
\begin{figure}[h!]
	\begin{minipage}{0.7\textwidth}
		\centering
		\includegraphics[width=\textwidth]{009DockerDesktop.png}
		\caption{Configurar Virtualización}
	\end{minipage}
	\label{fig:figure010}
\end{figure}

\begin{figure}[h!]
	\begin{minipage}{0.7\textwidth}
		\centering
		\includegraphics[width=\textwidth]{010DockerDesktop.png}
		\caption{Configurar Virtualización}
	\end{minipage}
	\label{fig:figure011}
\end{figure}

Tras configurar la virtualización y reiniciar, se puede ver que se ejecuta correctamente Docker Desktop
\begin{figure}[h!]	
	\begin{minipage}{0.7\textwidth}
		\centering
		\includegraphics[width=\textwidth]{011DockerDesktop.png}
		\caption{Docker Desktop en ejecución}
	\end{minipage}		
	\label{fig:figure012}
\end{figure}

Una vez instalado todo podemos empezar a ejecutar el servidor:

Se utiliza el comando : 
\begin{lstlisting}
docker build -t nginx-rtmp-server .
\end{lstlisting}
\begin{figure}[h!]
	\begin{minipage}{0.7\textwidth}
		\centering
		\includegraphics[width=\textwidth]{012BuildingProject.png}
		\caption{Construyendo el proyecto}
	\end{minipage}
	\label{fig:figure013}
\end{figure}

\begin{figure}[h!]
	\begin{minipage}{0.7\textwidth}
		\centering
		\includegraphics[width=\textwidth]{013BuildingProject.png}
		\caption{Construyendo el proyecto}
	\end{minipage}
	\label{fig:figure014}
\end{figure}

\begin{figure}[h!]
	\begin{minipage}{0.6\textwidth}
		\centering
		\includegraphics[width=\textwidth]{014BuildingProject.png}
		\caption{Proyecto construido}
	\end{minipage}
	\label{fig:figure015}
\end{figure}


Se levanta el servidor con la instrucción:

\begin{lstlisting}
docker run -d -p 1935:1935 -p 80:80 -p 443:443 --name nginx-stream --network streaming_network nginx-rtmp-server &&  docker logs -f nginx-stream
\end{lstlisting}



Puede dar un error si no está creada la red y se revisa con el siguiente comando:
\begin{lstlisting}
docker network create streaming_network
docker network ls
\end{lstlisting}

\begin{figure}[h!]
	\begin{minipage}{0.6\textwidth}
		\centering
		\includegraphics[width=\textwidth]{015RunningProject.png}
		\caption{Ejecutando el proyecto}
	\end{minipage}
	\label{fig:figure016}
\end{figure}

\begin{figure}[h!]
	\begin{minipage}{0.6\textwidth}
		\centering
		\includegraphics[width=\textwidth]{016RunningProject.png}
	\caption{Ejecutando el proyecto}
	\end{minipage}		
	\label{fig:figure017}
\end{figure}
\clearpage


\subsection{Servidor nginx HLS }

Los ficheros de configuración del servidor son los siguientes:
\begin{itemize}
	\item 	Carpeta certs: carpeta con los certificados para la ejecución protegida del servidor
	\item 	Carpeta Monitoring: carpeta con la configuración de prometheus y el docker-compose para la ejecución de todos los servicios. 
	\item 	Dockerfile: configuración de la imagen docker
	\item 	LICENSE licencia MIT
	\item 	main.html: pagina principal del servidor donde se puede conectar desde cualqueir navegador con acceso a la red del servidor
	\item 	nginx.conf: Configuración del servidor nginx
	\item 	README.md: información importante para configurar y desplegar el servidor	
\end{itemize}

\textbf{Revisar} ver si es mejor usar imágenes o texto para los ficheros del proyecto


\lstinputlisting[style=yaml]{servers/NginxRTPM/monitoring/docker-compose.yml}


\begin{figure}[h!]
	\begin{minipage}{0.7\textwidth}
		\centering
		\includegraphics[width=\textwidth]{019docker-compose00.png}
		\caption{docker-compose 1 de 3}
	\end{minipage}
	\label{fig:figure018}
\end{figure}

\begin{figure}[h!]
	\begin{minipage}{0.7\textwidth}
		\centering
		\includegraphics[width=\textwidth]{019docker-compose01.png}
		\caption{docker-compose 2 de 3}
	\end{minipage}		
	\label{fig:figure019}
\end{figure}

\begin{figure}[h!]
	\begin{minipage}{0.7\textwidth}
		\centering
		\includegraphics[width=\textwidth]{019docker-compose02.png}
		\caption{docker-compose 3 de 3}
	\end{minipage}		
	\label{fig:figure020}
\end{figure}


\end{document}