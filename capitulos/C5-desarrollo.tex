\documentclass[../main.tex]{subfiles}

\begin{document}

Se seguira el libro Mastering kubernetes (ToDo ) en su cuarta edicion. Para la creacion del cluster es necesario instalar Rancher desktop (https://docs.rancherdesktop.io/getting-started/installation/\#linux) que es una aplicacion de escritorio que permitira ejecutar Docker. Rancher Desktop instalara otras herramientas como: 

\begin{itemize}
	\item 	Helm
	\item 	Kubectl
	\item 	Nerdctl
	\item 	Moby (open source Docker)
	\item 	Docker Compose
\end{itemize}

Se instala pass, necesario para guardar credenciales que se pasan via docker login y nerdctl login.

Una vez instalado Rancher es recomendable generar una GPG key para securizar los secrets. para ello hacmoes lo siguiente:
\includegraphics{pass00.png}
\includegraphics{rancher00.png}

Se instala docker desktop siguiendo las instrucciones de la página oficial
https://docs.docker.com/desktop/setup/install/linux/ubuntu/

comprobamos que se han instalado correctamente tanto kubectl como docker
\includegraphics{dockerKubectlVerions.png}
\includegraphics{docker00.png}


el usuario debe tener permisos de docker y de video 

  sudo usermod -aG video $USER
    sudo usermod -aG docker $USER
    
    
    
    
    
Opción B:
Se utiliza una distribución basada en Ubuntu como es Mint en su versión XFCE que es una versión ligera y se realiza la instalación en una MV para comprobar que la instalación funciona
se deb instalar: 
Docker:
seguimos las instrucciones:
\href{https://docs.docker.com/engine/install/ubuntu/#install-using-the-repository}{Instalar Docker}.

\includegraphics{001 install Docker Desktop.png}
\includegraphics{002 install Docker Desktop.png}
\includegraphics{003 install Docker Desktop.png}
\includegraphics{004 install Docker Desktop.png}
kubectl
\href{https://kubernetes.io/docs/tasks/tools/install-kubectl-linux/#install-kubectl-binary-with-curl-on-linux}{Instalar Kubectl}.



\end{document}